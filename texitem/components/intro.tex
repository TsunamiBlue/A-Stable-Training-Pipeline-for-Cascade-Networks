\section{Introduction}
The deep neural network has got unparalleled success in Artificial Intelligence due to its generalization ability. Pre-defined network architecture and hyper-parameters are necessary before training classic neural networks algorithms, and the performance of neural networks like CNN are usually sensitive to these settings~\cite{CNNClassification2015}. Constructive model is a good way of estimate the problem complexity by the network complexity which is computationally economical~\cite{constructive}. However, constructive networks usually make use of greedy approach to find an appropriate network architecture~\cite{constructive}, the performance of the model may drop right after adding new initialised neurons. In this article we intend to use a modified version of Constructive Cascade Networks ~\cite{GeneralisationConstructiveCascade2009}, Generative Cascade Network Employing Progressive RPROP, or CVAE-Casper ~\cite{CASPER1997} which is inspired by ~\cite{CASPER1997} and ~\cite{cvae}, to solve two medical classification problem, and show that CVAE-Casper has more stable performance than vanilla Casper while training. \\
There are two datasets to validate our work. The first one is SARS-CoV-1 dataset. SARS-CoV-2, also known as COVID-19, has become a global pandemic disease from 2020 spring and also a landmark event in history. For the first dataset we are intended to build a classification deep learning model on the first SARS subspecies: SARS-CoV, which has more severe symptoms, higher fatality rate and lower transmission rate ~\cite{SARSCOV2_2020}. Since some physiological indicators like sequential body temperature are easy to measure, we will only use sequential body temperature for classification task. The secondly dataset is Cardiovascular disease dataset. The second classification task is more complex than the first one and the model will make use of physical examination data to predict Cardiovascular disease. We will analyze the stability of CVAE-Casper under both simple and complex dataset. The contributions we made in this article are as follows. Firstly, we introduced a new training method for constructive cascade network to both obtain stable performance while training and higher test accuracy while testing. Secondly, we introduced two classification model that can well predict SARS and Cardiovascular diseases which are common now.